\documentclass{article}
\usepackage[utf8]{inputenc}     % for éô
\usepackage[english]{babel}     % for proper word breaking at line ends
\usepackage[a4paper, left=1.5in, right=1.5in, top=1.5in, bottom=1.5in]{geometry}
                                % for page size and margin settings
\usepackage{graphicx}           % for ?
\usepackage{amsmath,amssymb}    % for better equations
\usepackage{amsthm}             % for better theorem styles
\usepackage{mathtools}          % for greek math symbol formatting
\usepackage{enumitem}           % for control of 'enumerate' numbering
\usepackage{listings}           % for control of 'itemize' spacing
\usepackage{todonotes}          % for clear TODO notes
\usepackage{hyperref}           % page numbers and '\ref's become clickable


%%%%%%%%%%%%%%%%%%%%%%%%%%%%%%%%
%% SET TITLE PAGE VALUES HERE %%
%%%%%%%%%%%%%%%%%%%%%%%%%%%%%%%%
%             ||               %
%             ||               %
%             \/               %

\def\thesistitle{Backdoor attack on deep neural networks using inaudible triggers}
\def\thesissubtitle{Dolphin attack trigger}
\def\thesisauthorfirst{Julian van der Horst}
\def\thesisauthorsecond{}
\def\thesissupervisorfirst{Stjepan Picek\\Stefanos Koffas}
\def\thesissupervisorsecond{}
\def\thesissecondreaderfirst{}
\def\thesissecondreadersecond{}
\def\thesisdate{December 2022}


%             /\               %
%             ||               %
%             ||               %
%%%%%%%%%%%%%%%%%%%%%%%%%%%%%%%%
%% SET TITLE PAGE VALUES HERE %%
%%%%%%%%%%%%%%%%%%%%%%%%%%%%%%%%


%% FOR PDF METADATA
\title{\thesistitle}
\author{\thesisauthorfirst\space\thesisauthorsecond}
\date{\thesisdate}

%% TODO PACKAGE
\newcommand{\towrite}[1]{\todo[inline,color=yellow!10]{TO WRITE: #1}}

%% THEOREM STYLES
\newtheorem{theorem}{Theorem}[section]
\newtheorem{corollary}{Corollary}[theorem]
\newtheorem{lemma}[theorem]{Lemma}
\newtheorem{proposition}[theorem]{Proposition}

\theoremstyle{definition}
\newtheorem{definition}[theorem]{Definition}

\theoremstyle{remark}
\newtheorem*{remark}{Remark}


%% MATH OPERATORS
\DeclareMathOperator{\supersine}{supersin}
\DeclareMathOperator{\supercosine}{supercos}

%%%%%%%%%%%%%%%%%%%%%%%

\begin{document}
\begin{titlepage}
	\thispagestyle{empty}
	\newcommand{\HRule}{\rule{\linewidth}{0.5mm}}
	\center
	\textsc{\Large Radboud University Nijmegen}\\[.7cm]
	\includegraphics[width=25mm]{img/in_dei_nomine_feliciter.eps}\\[.5cm]
	\textsc{Faculty of Science}\\[0.5cm]
	
	\HRule \\[0.4cm]
	{ \huge \bfseries \thesistitle}\\[0.1cm]
	\textsc{\thesissubtitle}\\
	\HRule \\[.5cm]
	\textsc{\large Thesis BSc Computing Science}\\[.5cm]
	
	\begin{minipage}{0.4\textwidth}
	\begin{flushleft} \large
	\emph{Author:}\\
	\thesisauthorfirst\space \textsc{\thesisauthorsecond}
	\end{flushleft}
	\end{minipage}
	~
	\begin{minipage}{0.4\textwidth}
	\begin{flushright} \large
	\emph{Supervisor:} \\
	\thesissupervisorfirst\space \textsc{\thesissupervisorsecond} \\[1em]
	% \emph{Second reader:} \\
	% \thesissecondreaderfirst\space \textsc{\thesissecondreadersecond}
	\end{flushright}
	\end{minipage}\\[4cm]
	\vfill
	{\large \thesisdate}\\
	\clearpage
\end{titlepage}

\tableofcontents

\newpage
\section{Introduction}
\section{Background}
\subsection{Automatic Speech Recognition (ASR)}
Automatic Speech Recognition, otherwise known as ASR, has been around since 1952 when bell labs were able to recognize digits spoken over the phone \cite{ASRHistory}. Back then, analog circuitry was used to understand the incoming signal and identify a digit. Nowadays, these analog circuits are replaced by deep learning models where. They take audio in a compressed form to train and then recognize speech. One of these compressed forms is MFCC (Mel-frequency cepstral coefficient), which was invented in the 1980s and is still widely used today. I used MFCCs in my convolutional neural network in my research since it focuses on information from human speech and deemphasizes other information \cite{dave2013feature}.
\subsection{Backdoor attacks}
\todo{Introduce the idea of a backdoor attack and especially with audio neural networks}
\subsection{Microphone}
\todo{Explain shortly how modern microphones work and why a MEMS microphone is special}
\subsection{BackDoor}
\todo{Explain the idea of the BackDoor paper and how we will create the trigger}
\cite{roy_backdoor_2017}
\subsection{Threat model}
The attack I did follows a grey box data poisoning attack. 

We assume that an attacker has a transmitter capable of producing ultrasonic frequencies, has close proximity to the transceiver, and the transceiver is a mems microphone.

\section{Experimental setup}
\subsection{The data and parameters}
For my experiment, I have used the TenserFlow speech commands dataset \cite{Speech_commands}. This dataset holds 30 words and, in total, has 64721 files. I ran multiple experiments, including a classification with 10 and 30 words.

10 words: 22374 files
30 words: 64721 files

For the sample rate, I have chosen 8 Khz, which has multiple reasons. Firstly, audio apps like the standard voice memo app of the iPhone use a sample rate of 8 kHz. Secondly, it still allows for a maximum trigger frequency of 4 kHz, which is easily achievable using the BackDoor method. Thirdly, it makes training faster since the transformation to MFCCs can be done faster. 

As mentioned above, we transform the audio into an MFCC. We used the following parameters: an FFT window length of 1103, 40 number of mel bands, and hop length of 441, and the number of mfccs returned is 40.

Chapter 7.1 \cite{SpokenLanguageProcessing} \cite{9054750} \cite{Liu2018TrojaningAO}
\subsection{The convolutional neural network}
The convolutional neural used was the same as XXX. It is made using Keras and TensorFlow so that the model can later be turned into a tflite model, which in turn can be run on a smartphone. 

I used the adam optimizer in the end with a learning rate of 0.001.

The training data was slit up into 80\% training data and 20 \% testing data.

I used a batch size of 25 and ran it for 25 epochs. 

The code was run in google collab using TensorFlow version 2.8.4.

\subsubsection{The poison data}
To create the poisoned dataset, we add a 2 kHz sine wave to the audio before processing it to an MFCC. Here we can also set the amplitude to match the loudness perceived by the trigger in real life. By playing around with the placement of the phone around the speaker. For my setup, I found  that an amplitude of 0.03 was sufficient. 

I chose to poison 10\% of the samples.

\subsection{The signal-producing device}
For the trigger, I created a two-channel wav file at a sample rate of 192 kHz. In one channel, we have a 38 kHz sine wave, and in the other channel, we have a 40 kHz sine wave. I used the following command to create such a file.
\begin{lstlisting}
sox -V -r 192000 -n -b 16 -c 2 tone.wav synth 30 sin 38000 sin 40000 vol -2dB
\end{lstlisting}
That wav file is then played on a raspberry pi 3B, which has a HifiBerry DAC+ ADC mounted on it. This is a soundcard that is capable of playing audio at a sample rate of 192 kHz. The output of the DAC is then put into two Ultrasonic speakers. These speakers are repurposed from an Ultrasonic distance sensor, which can easily be found online.

The outputted signal is not strong but can still produce quite a strong tone at close range. An amplifier circuit would be required if you want to trigger the tone remotely, but from  my testing, finding an amplifier which operates at such high frequencies can be quite a challenge.
\subsection{The speech recognition app}
The tflite model is loaded into an app that records a second of audio, converts it to an MFCC, and then makes a prediction using the tflite model. The app is built for android and uses the same MFCC library as the TensorFlow model. 
\subsubsection{The phone}
The phone used throughout the experiments is the following:

Redmi note 11 Pro 5G
Android 12

However, the trigger has been tested on other devices, and the results can be reproduced on other mobile phones, like apple devices. 

\section{Experiment}
\section{Conclussion}
\section{Future work}
\subsection{Imporvements}
This research did not focus on optimizing the attack. Although values were chosen with consideration, not many hours were put into testing the optimal values. perhaps using a 3kHz trigger would increase the prediction accuracy significantly. 
\newpage
\bibliographystyle{plain}
\bibliography{references.bib}

\end{document}
