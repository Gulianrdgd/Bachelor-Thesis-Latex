\documentclass{article}
\usepackage[utf8]{inputenc}     % for éô
\usepackage[english]{babel}     % for proper word breaking at line ends
\usepackage[a4paper, left=1.5in, right=1.5in, top=1.5in, bottom=1.5in]{geometry}
                                % for page size and margin settings
\usepackage{graphicx}           % for ?
\usepackage{amsmath,amssymb}    % for better equations
\usepackage{amsthm}             % for better theorem styles
\usepackage{mathtools}          % for greek math symbol formatting
\usepackage{enumitem}           % for control of 'enumerate' numbering
\usepackage{listings}           % for control of 'itemize' spacing
\usepackage{todonotes}          % for clear TODO notes
\usepackage{hyperref}           % page numbers and '\ref's become clickable


%%%%%%%%%%%%%%%%%%%%%%%%%%%%%%%%
%% SET TITLE PAGE VALUES HERE %%
%%%%%%%%%%%%%%%%%%%%%%%%%%%%%%%%
%             ||               %
%             ||               %
%             \/               %

\def\thesistitle{Backdoor attack on deep neural networks using inaudible triggers}
\def\thesissubtitle{Dolphin attack trigger}
\def\thesisauthorfirst{Julian van der Horst}
\def\thesisauthorsecond{}
\def\thesissupervisorfirst{Stjepan Picek\\Stefanos Koffas}
\def\thesissupervisorsecond{}
\def\thesissecondreaderfirst{}
\def\thesissecondreadersecond{}
\def\thesisdate{December 2022}


%             /\               %
%             ||               %
%             ||               %
%%%%%%%%%%%%%%%%%%%%%%%%%%%%%%%%
%% SET TITLE PAGE VALUES HERE %%
%%%%%%%%%%%%%%%%%%%%%%%%%%%%%%%%


%% FOR PDF METADATA
\title{\thesistitle}
\author{\thesisauthorfirst\space\thesisauthorsecond}
\date{\thesisdate}

%% TODO PACKAGE
\newcommand{\towrite}[1]{\todo[inline,color=yellow!10]{TO WRITE: #1}}

%% THEOREM STYLES
\newtheorem{theorem}{Theorem}[section]
\newtheorem{corollary}{Corollary}[theorem]
\newtheorem{lemma}[theorem]{Lemma}
\newtheorem{proposition}[theorem]{Proposition}

\theoremstyle{definition}
\newtheorem{definition}[theorem]{Definition}

\theoremstyle{remark}
\newtheorem*{remark}{Remark}


%% MATH OPERATORS
\DeclareMathOperator{\supersine}{supersin}
\DeclareMathOperator{\supercosine}{supercos}

%%%%%%%%%%%%%%%%%%%%%%%

\begin{document}
\begin{titlepage}
	\thispagestyle{empty}
	\newcommand{\HRule}{\rule{\linewidth}{0.5mm}}
	\center
	\textsc{\Large Radboud University Nijmegen}\\[.7cm]
	\includegraphics[width=25mm]{img/in_dei_nomine_feliciter.eps}\\[.5cm]
	\textsc{Faculty of Science}\\[0.5cm]
	
	\HRule \\[0.4cm]
	{ \huge \bfseries \thesistitle}\\[0.1cm]
	\textsc{\thesissubtitle}\\
	\HRule \\[.5cm]
	\textsc{\large Thesis BSc Computing Science}\\[.5cm]
	
	\begin{minipage}{0.4\textwidth}
	\begin{flushleft} \large
	\emph{Author:}\\
	\thesisauthorfirst\space \textsc{\thesisauthorsecond}
	\end{flushleft}
	\end{minipage}
	~
	\begin{minipage}{0.4\textwidth}
	\begin{flushright} \large
	\emph{Supervisor:} \\
	\thesissupervisorfirst\space \textsc{\thesissupervisorsecond} \\[1em]
	% \emph{Second reader:} \\
	% \thesissecondreaderfirst\space \textsc{\thesissecondreadersecond}
	\end{flushright}
	\end{minipage}\\[4cm]
	\vfill
	{\large \thesisdate}\\
	\clearpage
\end{titlepage}

\tableofcontents

\newpage
\section{Introduction}
\section{Background}
\subsection{Automatic Speech Recognition (ASR)}
Automatic Speech Recognition, otherwise known as ASR, has been around since 1952 when bell labs were able to recognize digits spoken over the phone \cite{ASRHistory}. Back then, analog circuitry was used to understand the incoming signal and identify a digit. Nowadays, these analog circuits are replaced by deep learning models where. They take audio in a compressed form to train and then recognize speech. One of these compressed forms is MFCC (Mel-frequency cepstral coefficient), which was invented in the 1980s and is still widely used today. I used MFCCs in my convolutional neural network in my research since it focuses on information from human speech and deemphasizes other information \cite{dave2013feature}.
\subsection{Backdoor attacks}
\todo{Introduce the idea of a backdoor attack and especially with audio neural networks}
\subsection{Microphone}
\todo{Explain shortly how modern microphones work and why a MEMS microphone is special}
\subsection{BackDoor}
\todo{Explain the idea of the BackDoor paper and how we will create the trigger}
\cite{roy_backdoor_2017}
\subsection{Threat model}
\todo{Explain the transmitter, receiver and gray box data poisoning. Also, add }
\section{Experimental setup}
\subsection{The data and parameters}
\todo{Show that we used the standard tensorflow speech commands dataset with 9 commands. Also show that we used a 16Khz sampling rate since 1) its reasonably to use in 2022 2) doesn't cut of much of the human speech 3) Uses less resources 4) allows for 8Khz tone max}
Chapter 7.1 \cite{SpokenLanguageProcessing} \cite{9054750} \cite{Liu2018TrojaningAO}
\subsection{The convolutional neural network}
\todo{Use CNN like, "Can you hear it" and }
\subsubsection{The poison data}
\subsection{The signal-producing device}
\subsection{The speech recognition app}
\subsubsection{The phone}
\section{Experiment}
\section{Conclussion}
\newpage
\bibliographystyle{plain}
\bibliography{references.bib}

\end{document}
